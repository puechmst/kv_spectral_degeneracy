
\section{Introduction}
Koszul–Vinberg algebras (also known as pre-Lie algebras) naturally arise in differential geometry through the use of torsion-free affine connections. When equipped with a gauge structure $(M, \nabla)$, the space of vector fields on a smooth manifold inherits a pre-Lie algebra structure defined by the product $X \circ Y = \nabla_X Y$. Unlike Lie algebras, this structure does not satisfy the Jacobi identity, but instead encodes curvature information through the associator.

In this paper, we study the interaction between pre-Lie algebras, gauge-theoretic structures, and symmetric $(1,1)$-tensor fields that satisfy a compatibility condition known as the gauge equation:

$$
\nabla \theta = \theta \nabla^*,
$$

where $\nabla$ and $\nabla^*$ are torsion-free affine connections dual with respect to a Riemannian metric $g$. This equation governs how $\theta$ mediates between two flat (or nearly flat) geometric structures and plays a central role in the construction of differential complexes arising from Koszul–Vinberg (KV) algebras. Furthermore, we study a distinct KV-algebra structure defined not by the connection $\nabla$, but by a symmetric (1,2)-tensor field $D$, defined as the difference between two torsion-free connections: $D := \nabla^* - \nabla$. We consider the product:

$X \circ Y := D(X, Y),$

which is symmetric and defines a valid pre-Lie algebra when $D$ satisfies certain compatibility conditions. In particular, we explore the associated cohomological structure and spectral sequence induced by this KV product.

The central goal is to identify conditions under which the spectral sequence associated to the cohomology of this KV algebra degenerates early. We show that when a symmetric tensor field $\theta$ satisfies a gauge-type condition adapted to $D$, the resulting spectral sequence degenerates at a low stage.

We work within the setting of flat coKähler manifolds, where the underlying geometric structures—such as the Reeb field $\xi$, the contact form $\eta$, and the endomorphism $\phi$—induce a natural splitting of the tangent bundle and allow for explicit computations. In this context, we reinterpret classical gauge equations and deformation tensors relative to the new KV-algebra structure based on $D$, not $\nabla$.

The paper is organized as follows: Section 2 introduces the general theory of KV algebras and their cohomology, adapted to the case where the product is defined by a symmetric tensor $D$. Section 3 introduces a generalized gauge equation and explores conditions under which cochains and higher-order obstructions vanish. Section 4 focuses on flat coKähler manifolds and presents explicit computations showing the degeneracy of the spectral sequence associated to the $D$-based KV algebra.

This reinterpretation of KV-algebra structures and their cohomology reveals a new class of degeneracy results governed by the interplay between geometry (via the coKähler structure), algebra (via $D$), and analysis (via the gauge tensor $\theta$).

This work contributes to the understanding of the geometric meaning behind the gauge equation in differential geometry and its algebraic consequences in the deformation theory and cohomology of pre-Lie algebras.
