\section{The gauge equation}
Let $E$ be a vector bundle on a smooth manifold $M.$ As usual, $\Gamma_E(U)$ denotes the $C^\infty(U)$-module of its smooth sections over an open set $U \subset M.$ Given a pair $V \subset U \subset M$ of open sets, the inclusion map $i_{VU} \colon V \hookrightarrow U$ gives rise to a map of modules $\Gamma_E(i_{VU}) \colon \Gamma_E(U) \to \Gamma_E(V).$ The functor $\Gamma_E$ defined that way is a sheaf, called the sheaf of smooth sections of $E.$
It is well known \cite{husemoller1994} that if $\nabla$ is a Koszul connection and $\theta$ is an invertible $(1,1)$-tensor, then $\tilde{\nabla} = \theta \nabla \theta^{-1}$ is a Koszul connection.
Rewriting the above relation gives:
\begin{equation}
    \label{eq:invertible_gauge_eq}
    \theta \nabla = \tilde{\nabla} \theta.
\end{equation}
The form \ref{eq:invertible_gauge_eq} is valid even if $\theta$ is not invertible, thus it makes sense to introduce the general gauge equation.
\begin{defn}
Let $\left(\nabla^1,\nabla^2\right)$ a pair of Koszul connections. A $(1,1)$-tensor is said to satisfy the gauge equation for $\left(\nabla^1,\nabla^2\right)$ if:
\begin{equation}
\label{eq:gauge_eq}
\nabla^2 \theta = \theta \nabla^1.
\end{equation}
\end{defn}
    The gauge equation is compatible with the sheaf $\Gamma_E.$ Let $U \subset M$ be an open set, $f \in C^\infty(U),$ and $s \in \Gamma_E(U)$ such that $\nabla^2 \theta s = \theta \nabla^1 s.$
    Then:
    \begin{equation}
        \label{eq:sheaf_gauge_equation}
        \nabla^2 \theta f s = \nabla^2 f \theta s  = \nabla^2(f) \theta s + f \nabla^2 \theta s 
        = \nabla^1(f) \theta s  + f \theta \nabla^1 s 
         = \theta \nabla^1 f s.
    \end{equation}
It thus make sense to speak about the sheaf of solutions of the gauge equation for $\left(\nabla^1, \nabla^2 \right).$
Finally, let $\left(E, \nabla^1 \right), \left(F, \nabla^2 \right)$ be gauge structures with $E,F$ bundles on $M$. Koszul connections $\nabla^1,\nabla^2$ can be interpreted as exterior derivatives on (sheaves of) 0-forms. Given a bundle morphism $\theta \colon  E \to F,$ $\theta$ is said to be a solution of the gauge equation if the next diagram commutes:
\begin{equation}
    \label{eq:koszul_exterior_der}
    \begin{tikzcd}
    \Gamma_E \rar["\nabla^1"] \dar["\theta"] & \Gamma_{T^\star M  \otimes E} \dar["\Id \otimes \theta"]\\
    \Gamma_F \rar["\nabla^2"] & \Gamma_{T^\star M \otimes F}
    \end{tikzcd}
\end{equation}
This general definition obviously agrees with the previous one. The identity morphism $\Id_E \colon E \to E$ satisfies the gauge equation $\Id \nabla = \nabla \Id$ for any gauge structure $\left(E,\nabla\right)$ and the composition of two bundle morphisms satisfying the gauge equation satisfies again the gauge equation for if $\nabla^2 \theta = \theta \nabla^1$ and $\nabla^3 \eta = \eta \nabla^2$, then:
\[
\eta \theta \nabla^1 = \eta \nabla^2 \theta = \nabla^3 \eta \theta.
\]
There is thus a category $\gaugecat(M)$ whose objects are gauge structures and morphisms given by solutions of the gauge equation.
The next proposition is merely an application of definition \ref{eq:gauge_eq}.
\begin{prop}
\label{prop:conjugate_connections}
Let the triple $\left(\nabla_1, \nabla_2,\theta \right)$ be a solution of the gauge equation $\nabla_1 \theta = \theta \nabla_2.$ 
For any couple $(U,V)$ of bundle automorphism on $E$ (resp.L $F$) the triple:
\[
\left( U \nabla_1 U^{-1}, U \theta V^{-1}, V \nabla_2 V^{-1} \right)
\]
is a solution of a gauge equation.
\end{prop}
As usual, everything can be made local, so basic linear algebra shows that in a suitable local frame, $\theta$ can be reduced to a block diagonal form:
\begin{equation}
    \label{eq:reduced_theta}
    \theta = \left( \begin{array}{c|c} \text{Id} & 0 \\ \hline
    0 & 0
    \end{array} \right)
\end{equation}
When the manifold $M$ is equipped with a Riemannian metric, there exists a dual connection associated to an arbitrary Koszul connection. 
\begin{defn}
    \label{def:conjugate_connection}
    Let $\nabla$ be an affine connection. Its dual with respect to $g$ is the connection $\nabla^\star$ defined by the relation:
    \begin{equation}
    \label{eq:conjugate_connection}
    \forall Z \in TM, \forall r,s \in \Gamma \left(TM 
    \right), Z\left( g(r,s) \right) = g\left(\nabla_Z r, s\right) + 
    g\left( r, \nabla_Z^\star s\right)
    \end{equation}
\end{defn}

\begin{defn}
\label{def:endo_conjugate}
Let $\theta$ be a bundle morphism on $TM.$ Its (Riemannian) conjugate,
denoted $\theta^\star$, is the bundle morphism defined by:
\begin{equation}
    \label{eq:endo_conjugate}
    \forall X,Y \in TM, \, g\left( \theta X, Y\right) = 
    g\left( X, \theta^\star Y\right)
\end{equation}
\end{defn}
\begin{prop}
\label{prop:inverse_unitary}
If $U \colon TM \to TM$ is a unitary bundle isomorphism, that is:
\[
\forall X,Y \in TM, \, g\left( UX, UY \right) = g(X,Y)
\]
Then $U^{-1} = U^\star.$
\end{prop}

\begin{prop}
\label{prop:unitary_conjugate}
For any Koszul connection $\nabla$ and unitary bundle isomorphism $U$ :
\begin{equation}
\label{eq:unitary_conjugate}
\left(U \nabla U^\star\right)^\star = U \nabla^\star U^\star
\end{equation}
\end{prop}
The proof can be found in \cite{boyom2024}.
\begin{prop}[\cite{boyom2024}]
\label{prop:unitary_conjugate_gauge}
If the triple $\left(\nabla,\nabla^+,\theta \right)$ satisfies the gauge equation $\nabla \theta = \theta \nabla^+$, so does  $\left(U \nabla, U^+, U \nabla^+ U^+, U^+ \theta U\right)$ for any unitary isomorphism $U.$
\end{prop}
\begin{rem}
If $\theta$ is normal and satisfies the gauge equation, it is reducible locally to the form
\begin{equation}
    \label{eq:reduced_normal_theta}
    \theta = \left( \begin{array}{c|c} \text{Id} & 0 \\ \hline
    0 & 0
    \end{array} \right)
\end{equation}
by unitary bundle automorphisms.
\end{rem}
\begin{prop}
\label{prop:parallel_g_tensor}
Let the triple $\left(\nabla,\nabla^+,\theta \right)$ satisfy the gauge equation $\nabla \theta = \theta \nabla^+.$ Then the tensor:
\begin{equation}
    \label{eq:parallel_g_tensor}
    g_\theta \colon (X,Y) \mapsto g\left( \theta X, Y\right)
\end{equation}
is $\nabla$ parallel.
\end{prop}

Proposition  \ref{prop:parallel_g_tensor} shows that $TM$ can be written as two direct sums:
\begin{equation}
    \label{eq:split_tm}
    TM = \ker \theta \oplus \im \theta , \, TM = \ker \theta^+ \oplus \im \theta^+.
\end{equation}
It is clear from proposition \ref{prop:parallel_g_tensor} that if $\theta$ is symmetric, that is $\theta = \theta^+$, the tensor:
\begin{equation}
(X,Y) \mapsto \frac{1}{2} g\left( \theta X, Y\right) + g\left( \theta X, Y\right)
\end{equation}
is $\nabla$-parallel.
When $\theta$ is skew symmetric $\theta = -\theta^+$, the same is true for:
\begin{equation}
(X,Y) \mapsto \frac{1}{2} g\left( \theta X, Y\right) - g\left( \theta X, Y\right)
\end{equation}
In the spirit of the definition of $\gaugecat(M)$, it makes sense to look after a category where morphisms are bundle morphisms $\theta$ such that dual gauge structures $(M,\nabla), (M, \nabla^\star)$ are related if $\theta \nabla = \nabla^\star \theta.$ Things are a little bit more intricate than in the general case, so some extra assumptions must be made.
\begin{defn}
    \label{def:flat_morphism}
    The $\flat \colon TM \to T^\star M$ isomorphism is defined by the relation:
    \begin{equation}
        \label{eq:flat_morphism}
        X^\flat(Y) = g(X, Y), X, Y \in TM.
    \end{equation}
\end{defn}
\begin{defn}
\label{def:partial_iso}
Let $E,F$ be two vector bundles on $M$ equipped with respective Riemannian metrics $g_E,g_F.$ A partial isometry from $E$ to $F$ is a bundle morphism $U$ such that the following diagram commutes:
\begin{equation}
    \label{eq:partial_iso}
    \begin{tikzcd}
        E \ar[r,"{}^\flat"] \ar[d,"U"] & E^\star \\
        F \ar[r,"{}^\flat"] & F^\star \ar[u,"U^\star"]
    \end{tikzcd}
\end{equation}
\end{defn}
\begin{rem}
    Definition \ref{def:partial_iso} is equivalent to the fact that for any $s,s^\prime \in \Gamma(E)$:
    \[
    g_F\left(Us, Us^\prime\right) = g_E\left(s,s^\prime\right).
    \]
\end{rem}
\begin{defn}
\label{def:general_dual_connections}
Let $U \colon E \to F$ be a partial isometry and $\nabla_2$ be a Koszul connection on $F$. Its dual $\nabla_2^+$ is the connection on $E$ defined by the relation
\begin{equation}
    \label{eq:general_dual_connections}
    U \nabla_2^+ = \nabla_2 U.
\end{equation}
\end{defn}
\begin{defn}
    \label{def:category_gauge_dual}
    The category $\gaugecatu(M)$ has objects $(E,\nabla)$ where $\nabla$ is a Koszul connection on $E$ and morphisms $(U, \theta) \colon (E, \nabla_1) \to (F, \nabla_2)$ where $U \colon E \to F$ is a partial isometry, $\theta \colon E \to F$ is a bundle morphism and $\nabla_1 = \nabla_2^+, \nabla_2 \theta = \theta \nabla_1.$
\end{defn}
The next two examples are extracted from \cite{boyom2024}. 
\begin{ex}
    Take $M = \R^2$ and consider the symplectic 2-form:
    \begin{equation}
        \omega \colon (x,y) \mapsto \exp{x} dx \wedge dy.
    \end{equation}
    Let $\nabla$ be a Koszul connection such that $\nabla \omega = 0$, $g$ an arbitrary Riemannian metric on $\R^2$ and $\nabla^+$ the dual of $\nabla$ with respect to $g$. Finally, let $\theta$ be the unique $(1,1)$-tensor such that, for all vector fields $X,Y$:
    \begin{equation}
        \omega(X,Y) = g(\theta X, Y)
    \end{equation}
    Then:
    \begin{equation}
        \nabla^+_X \theta Y = \theta \nabla_X Y. 
    \end{equation}
\end{ex}
\begin{ex}
    Take $M=\mathbb{S}^3$, $\nabla$ a torsion-less connection and $g$ a Riemannian metric. Any solution $\theta$ to the gauge equation:
    \begin{equation}
        \nabla^+_X \theta Y = \theta \nabla_X Y. 
    \end{equation}
    is either 0 or invertible.
\end{ex}

\begin{prop}
    \label{prop:gen_nijenhuis}
    Let $\theta$ be normal and satisfying the gauge equation. Let $D=\nabla^*-\nabla$. Then the tensor:
    \begin{equation}
        \label{eq:gen_nijenhuis}
        \begin{split}
        T_\theta \colon (X,Y) \mapsto & \left[ \theta X, \theta^*Y \right] - \theta^* \left[ \theta X, Y \right] 
        - \theta \left[X, \theta^*Y  \right] + \theta^* \theta \left[ X,Y \right]\\
        & +\theta^* D\left( \theta X, Y \right) - \theta D\left( X, \theta^*Y \right)
        \end{split}
    \end{equation}
    vanishes identically.
\end{prop} 
\begin{proof}
    We must first prove the tensoriality. Let $f \colon M \to \R$ be a smooth function. Since $D$ is known to be 
    a tensor, there is no need to take the last two terms into account. A direct computation shows that:
    \begin{equation}
        T_\theta(fX,Y) = f T_\theta(X,Y) +
         \left( - \theta^* Y(f) + \theta^* Y(f) + \theta \theta^* Y(f) - \theta^* \theta Y(f) \right) T_\theta(X,Y)
    \end{equation}
    Since $\theta$ is normal, $ T_\theta(fX,Y) = f T_\theta(X,Y) $.
    Now, a similar computation yields:
    \begin{equation}
        T_\theta(X,fY) = f T_\theta(X,Y) + 
        \left( \theta X(f) - \theta^*\theta X(f) - \theta  X(f) + \theta^* \theta X(f) \right)T_\theta(X,Y) = 0
    \end{equation}
    proving that $T_\theta$ is a tensor.

    $\nabla^*$ being without torsion:
    \begin{equation}
        \begin{split}
            \left[ \theta X, \theta^* Y \right] & = \nabla^*_{\theta X} \theta^* Y - \nabla^*_{\theta^*Y} \theta X \\
            & = \theta^* \nabla_{\theta X} Y - \theta \nabla_{\theta^*Y} X \\
            & = \theta^* \left( \nabla_Y \theta X + \left[ \theta X, Y \right] \right) 
            - \theta \left( \nabla_X \theta^* Y  + \left[ \theta^*Y, X \right]\right) \\
        \end{split}
    \end{equation}
    Introducing the difference tensor $D \colon (X,Y) \mapsto \nabla_X^* Y - \nabla_X Y$:
    \begin{equation}
       \begin{split}
         \left[ \theta X, \theta^* Y \right]  =&
           \theta^* \theta \nabla^*_Y X - \theta \theta^* \nabla^*_X Y  - \theta^* D\left( \theta X, Y \right) + \theta D\left( X, \theta^*Y \right)\\
           & + \theta^* \left[ \theta X, Y \right] + \theta \left[ X, \theta^* Y \right]
       \end{split} 
    \end{equation}
    Since $\theta$ is normal, $\theta^* \theta = \theta \theta^*$, thus:
    \begin{equation}
        \left[ \theta X, \theta^* Y \right] = \theta^* \theta \left[ Y,X \right] + \theta^* \left[ \theta X, Y \right] + \theta \left[ X, \theta^* Y \right]
        -\theta^* D\left( \theta X, Y \right) + \theta D\left( X, \theta^*Y \right)
    \end{equation}
    and the claim follows.
\end{proof}
\begin{rem}
    The tensoriality of $T_\theta$ depends critically on the normality of $\theta$. This was already pointed out by 
    Nijenhuis.
\end{rem}
\begin{rem}
    $T_\theta$ is the Nijenhuis tensor when $\theta$ is skew-symmetric and $\nabla = \nabla^*$. When $\theta$ is normal and satisfies the gauge equation,
    the above results shows that $-\theta^*D\left(  \theta X, Y\right)+\theta D\left( X,\theta^*Y \right)$ is the Nijenhuis tensor. 
\end{rem}