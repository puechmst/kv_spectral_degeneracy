\section{coK\"{a}hler manifolds}
CoK\"{a}hler manifolds form a class of almost contact metric manifolds closely related to both K\"{a}hler and cosymplectic geometry. Let $(M^{2n+1}, \phi, \xi, \eta, g)$ be an almost contact metric manifold, where:
\begin{enumerate}
\item $\phi$ is a (1,1)-tensor field,
\item $\xi$ is the Reeb (characteristic) vector field,
\item $\eta$ is the contact 1-form,
\item $g$ is a Riemannian metric satisfying:

  $$
  \eta(\xi) = 1,\quad \phi^2 = -\mathrm{Id} + \eta \otimes \xi,\quad g(\phi X, \phi Y) = g(X, Y) - \eta(X)\eta(Y).
  $$
\end{enumerate}
A coK\"{a}hler structure is defined by the following conditions:
\begin{enumerate}
\item $\nabla \phi = 0$,
\item $\nabla \eta = 0$, and
\item $d\eta = 0$, $d\Phi = 0$, where $\Phi(X,Y) = g(X, \phi Y)$.
\end{enumerate}
These conditions imply that $\phi$, $\eta$, and $g$ are all parallel with respect to the Levi-Civita connection $\nabla$, and that $(M, \phi, \xi, \eta, g)$ is both normal and cosymplectic. Importantly, the almost contact structure induces a foliation by leaves orthogonal to $\xi$, each of which inherits a K\"{a}hler structure from $\phi$ and $g$.

This leads to a canonical gauge structure on coK\"{a}hler manifolds, making them suitable for the application of KV-algebra techniques. Specifically, the Levi-Civita connection $\nabla$ on a coK\"{a}hler manifold is torsion-free and preserves the underlying structure tensors, allowing one to define pre-Lie products via

$$
X \circ Y := \nabla_X Y
$$

for vector fields $X, Y \in \Gamma(TM)$. The curvature conditions further imply symmetries in the associator that reflect pre-Lie algebra identities.

In particular, the canonical splitting $TM = \mathcal{D} \oplus \langle \xi \rangle$, where $\mathcal{D} = \ker(\eta)$, allows one to study the restriction of the Koszul-Vinberg algebra to the K\"{a}hler leaves and analyze the corresponding spectral sequence. These foliated structures are flat in transverse directions and lead naturally to vanishing torsion tensors $T_\theta$ when considering natural normal fields such as $\phi$. Thus, coK\"{a}hler manifolds provide an explicit class of geometric models where the degeneracy of the KV spectral sequence can be investigated under controlled curvature and holonomy assumptions.

Let $M = S^1 \times \mathbb{R}^{2n}$ with coordinates $(t, x^1, y^1, \dots, x^n, y^n)$. Define the flat product metric:

$$
g = dt^2 + \sum_{j=1}^n \left( (dx^j)^2 + (dy^j)^2 \right)
$$

Define a coK\"{a}hler structure:
\begin{itemize}
\item $\xi = \partial_t;$
\item $\eta = dt;$
\item $\phi(\partial_{x^j}) = \partial_{y^j}, \; \phi(\partial_{y^j}) = -\partial_{x^j}, \; \phi(\partial_t) = 0.$
\end{itemize}


Let $\nabla$ be the flat connection. Define the dual connection $\nabla^*$ as:

$$
\nabla^*_X Y = \nabla_X Y + D(X, Y)
$$

where $D$ is a symmetric (1,2)-tensor field such that:

$$
g(D(X,Y),Z) = -g(Y,D(X,Z)).
$$


Define $\theta$ diagonally:

$$
\theta = \lambda_0 \, \partial_t \otimes dt + \sum_{j=1}^n \left( \lambda_j \, \partial_{x^j} \otimes dx^j + \mu_j \, \partial_{y^j} \otimes dy^j \right)
$$

This $\theta$ is symmetric by construction.


Given $\nabla \theta = 0$, the gauge equation reduces to:

$$
0 = \theta D(X, \cdot).
$$
\begin{ex}
In dimension 2, let $A(\partial_x, \partial_x) = \partial_y$. Then $\theta(\partial_y) = 0$ is required. Thus, $\mu = 0$ is forced.
\end{ex}
\begin{ex}
On $M = S^1 \times \mathbb{R}^4$ we choose coordinates 
$$
(t, x^1, y^1, x^2, y^2)
$$.
We define metric
$$
g = dt^2 + (dx^1)^2 + (dy^1)^2 + (dx^2)^2 + (dy^2)^2,
$$
and the following coK\"{a}hler structure. Reeb vector field is $\xi = \partial_t$, a contact one-form $\eta = dt$, and structure tensor $\phi$:
$$
  \begin{aligned}
  \phi(\partial_{x^1}) &= \partial_{y^1}, & \phi(\partial_{y^1}) &= -\partial_{x^1}, \\
  \phi(\partial_{x^2}) &= \partial_{y^2}, & \phi(\partial_{y^2}) &= -\partial_{x^2}, \\
  \phi(\partial_t) &= 0.
  \end{aligned}
  $$
Let $\nabla$ denote a flat, torsionless connection on $M$. We now define tensor
$$
D(X, Y) := \phi(Y) \eta(X) + \phi(X) \eta(Y).
$$
This $D$ is symmetric and compatible with the coKähler structure.\\
Now, the dual connection (flat and torsionless) of $\nabla$ with respect to $g$ is defined by 
$$\nabla^*:=\nabla + D.$$
We define a symmetric tensor $\theta$:
$$
\theta = \lambda_0 \, \partial_t \otimes dt
+ \lambda_1 \, \partial_{x^1} \otimes dx^1
+ \mu_1 \, \partial_{y^1} \otimes dy^1
+ \lambda_2 \, \partial_{x^2} \otimes dx^2
+ \mu_2 \, \partial_{y^2} \otimes dy^2
$$

This $\theta$ is symmetric with respect to $g$, and constant in the coordinate basis.\\
We now check:

$$
\nabla \theta = \theta \nabla^* = \theta(\nabla + D)
\Rightarrow \nabla \theta = \theta D
$$

\end{ex}

