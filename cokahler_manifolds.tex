\section{coK\"{a}hler manifolds}
CoK\"{a}hler manifolds form a class of almost contact metric manifolds closely related to both K\"{a}hler and cosymplectic geometry. Let $(M^{2n+1}, \phi, \xi, \eta, g)$ be an almost contact metric manifold, where:
\begin{enumerate}
\item $\phi$ is a (1,1)-tensor field,
\item $\xi$ is the Reeb (characteristic) vector field,
\item $\eta$ is the contact 1-form,
\item $g$ is a Riemannian metric satisfying:

  $$
  \eta(\xi) = 1,\quad \phi^2 = -\mathrm{Id} + \eta \otimes \xi,\quad g(\phi X, \phi Y) = g(X, Y) - \eta(X)\eta(Y).
  $$
\end{enumerate}
A coK\"{a}hler structure is defined by the following conditions:
\begin{enumerate}
\item $\nabla \phi = 0$,
\item $\nabla \eta = 0$, and
\item $d\eta = 0$, $d\Phi = 0$, where $\Phi(X,Y) = g(X, \phi Y)$.
\end{enumerate}
These conditions imply that $\phi$, $\eta$, and $g$ are all parallel with respect to the Levi-Civita connection $\nabla$, and that $(M, \phi, \xi, \eta, g)$ is both normal and cosymplectic. Importantly, the almost contact structure induces a foliation by leaves orthogonal to $\xi$, each of which inherits a K\"{a}hler structure from $\phi$ and $g$.

This leads to a canonical gauge structure on coK\"{a}hler manifolds, making them suitable for the application of KV-algebra techniques. Specifically, the Levi-Civita connection $\nabla$ on a coK\"{a}hler manifold is torsion-free and preserves the underlying structure tensors, allowing one to define pre-Lie products via

$$
X \circ Y := \nabla_X Y
$$

for vector fields $X, Y \in \Gamma(TM)$. The curvature conditions further imply symmetries in the associator that reflect pre-Lie algebra identities.

In particular, the canonical splitting $TM = \mathcal{D} \oplus \langle \xi \rangle$, where $\mathcal{D} = \ker(\eta)$, allows one to study the restriction of the Koszul-Vinberg algebra to the K\"{a}hler leaves and analyze the corresponding spectral sequence. These foliated structures are flat in transverse directions and lead naturally to vanishing torsion tensors $T_\theta$ when considering natural normal fields such as $\phi$. Thus, coK\"{a}hler manifolds provide an explicit class of geometric models where the degeneracy of the KV spectral sequence can be investigated under controlled curvature and holonomy assumptions.

Let $M = S^1 \times \mathbb{R}^{2n}$ with coordinates $(t, x^1, y^1, \dots, x^n, y^n)$. Define the flat product metric:

$$
g = dt^2 + \sum_{j=1}^n \left( (dx^j)^2 + (dy^j)^2 \right)
$$

Define a coK\"{a}hler structure:
\begin{itemize}
\item $\xi = \partial_t;$
\item $\eta = dt;$
\item $\phi(\partial_{x^j}) = \partial_{y^j}, \; \phi(\partial_{y^j}) = -\partial_{x^j}, \; \phi(\partial_t) = 0.$
\end{itemize}


Let $\nabla$ be the flat connection. Define the dual connection $\nabla^*$ as:

$$
\nabla^*_X Y = \nabla_X Y + D(X, Y)
$$

where $D$ is a symmetric (1,2)-tensor field such that:

$$
g(D(X,Y),Z) = -g(Y,D(X,Z)).
$$


Define $\theta$ diagonally:

$$
\theta = \lambda_0 \, \partial_t \otimes dt + \sum_{j=1}^n \left( \lambda_j \, \partial_{x^j} \otimes dx^j + \mu_j \, \partial_{y^j} \otimes dy^j \right)
$$

This $\theta$ is symmetric by construction.


Given $\nabla \theta = 0$, the gauge equation reduces to:

$$
0 = \theta D(X, \cdot).
$$
\begin{ex}
In dimension 2, let $A(\partial_x, \partial_x) = \partial_y$. Then $\theta(\partial_y) = 0$ is required. Thus, $\mu = 0$ is forced.
\end{ex}
\begin{ex}
Let $(M, \varphi, \xi, \eta, g)$ be a flat coKähler manifold with local orthonormal frame $\{\xi, E_2, \ldots, E_n\}$ satisfying $\eta(\xi) = 1$, $\eta(E_j) = 0$, and let $\nabla$ denote the flat, torsion-free Levi-Civita connection. Define the symmetric $(1,1)$-tensor
\[
\theta = a(t)\, \eta \otimes \xi + \sum_{j=2}^n b_j(t)\, E^j \otimes E_j,
\]
where $E^j$ are the dual $1$-forms of $E_j$, and $\theta$ is time-dependent only. Assume $E_j$ are parallel along $\xi$, i.e., $\nabla_\xi E_j = 0$.

We verify the gauge-type condition
\[
(\nabla_X \theta)(Y) = \eta(Y)\theta(X) + \eta(X)\theta(Y),
\]
for all vector fields $X, Y$ on $M$.

\begin{itemize}
\item For $X = Y = \xi$:
\[
(\nabla_\xi \theta)(\xi) = \nabla_\xi(\theta(\xi)) - \theta(\nabla_\xi \xi) = \nabla_\xi(a(t)\xi) = a'(t)\xi.
\]
On the other hand, the right-hand side gives $2a(t)\xi$. Therefore,
\[
a'(t) = 2a(t) \quad \Rightarrow \quad a(t) = A e^{2t}.
\]

\item For $X = \xi$, $Y = E_j$:
\[
(\nabla_\xi \theta)(E_j) = \nabla_\xi(b_j(t) E_j) = b_j'(t) E_j,
\]
\[
\eta(E_j) = 0, \quad \eta(\xi) = 1 \Rightarrow \eta(Y)\theta(\xi) + \eta(X)\theta(Y) = \theta(E_j).
\]
Thus,
\[
b_j'(t) = b_j(t) \quad \Rightarrow \quad b_j(t) = B_j e^t.
\]

\item For $X = E_k$, $Y$ arbitrary:
The gauge condition becomes:
\[
(\nabla_{E_k} \theta)(Y) = \eta(Y)\theta(E_k).
\]
Compute the left-hand side:
\[
(\nabla_{E_k} \theta)(Y) = \nabla_{E_k}(\theta(Y)) - \theta(\nabla_{E_k} Y).
\]
For $Y = \xi$:
\[
(\nabla_{E_k} \theta)(\xi) = E_k(a)\xi + a \nabla_{E_k} \xi - \theta(\nabla_{E_k} \xi).
\]
Equating both sides:
\[
E_k(a)\xi + (a - \theta)\nabla_{E_k} \xi = \theta(E_k).
\]
Thus, either $\nabla_{E_k} \xi = 0$ and $E_k(a) = 0$ (implying $\theta(E_k) = 0$), or $\theta$ must be chosen to satisfy:
\[
\theta(E_k) = E_k(a)\xi + (a - \theta)(\nabla_{E_k} \xi).
\]
\end{itemize}

This example demonstrates how exponential time-dependent components of $\theta$ can satisfy the gauge-type condition on flat coKähler manifolds under appropriate geometric constraints.
\end{ex}
\begin{ex}

We now present a nontrivial example where the tensor field $\theta$ depends polynomially on the coordinate $t$ and satisfies the gauge equation $\nabla \theta = \theta \nabla^*$, but only after modifying the background geometry. This illustrates the necessity of adjusting the difference tensor $D$ when seeking polynomial solutions in $\theta$.

Let $M = S^1 \times \mathbb{R}^{2n}$ with coordinates $(t, x_1, y_1, \ldots, x_n, y_n)$ and flat coKähler structure as in previous examples. Define the symmetric $(1,1)$-tensor field
\[
\theta = a(t)\, \eta \otimes \xi + \sum_{j=1}^n \left( b_j(t)\, dx_j \otimes \partial_{x_j} + c_j(t)\, dy_j \otimes \partial_{y_j} \right),
\]
where the coefficient functions are taken to be polynomials in $t$, e.g., $a(t) = A t^k$, $b_j(t) = B_j t^{\ell_j}$, $c_j(t) = C_j t^{m_j}$ for constants $A, B_j, C_j$ and non-negative integers $k, \ell_j, m_j$.

Assume the connection $\nabla$ is flat and torsion-free, so that $\nabla \theta = \partial_t \theta$. In the previous examples, a constant $\theta$ sufficed to make $\nabla \theta = 0$ and the gauge equation reduced to $\theta D = 0$. However, for this variable $\theta$, the equation $\nabla \theta = \theta D$ becomes
\[
\frac{d}{dt} \theta = \theta D(\partial_t, \cdot),
\]
which imposes strong constraints on $D$.

To satisfy this equation, we define a non-flat difference tensor:
\[
D(X,Y) := \eta(X)\, \varphi(Y) + t \cdot \varphi(X)\, \eta(Y).
\]
We compute:
\[
\partial_t b_j(t) = t \cdot b_j(t) \quad \Rightarrow \quad b_j(t) = B_j \exp\left(\frac{t^2}{2}\right),
\]
so true polynomial solutions do not arise. However, for $b_j(t) = B_j t$, we find that:
\[
\partial_t b_j(t) = B_j = t \cdot b_j(t) \quad \Leftrightarrow \quad t = 1,
\]
which is only valid on a slice of $M$. Thus, a  polynomial solution is only compatible with the gauge equation if $D$ is choosen. 
\end{ex}

In the statistical coKähler setting we define, the structure tensor $\varphi$ is shown to satisfy the gauge equation $\nabla \varphi = \varphi \nabla^*$. This is a nontrivial and highly constrained equation in general, but here it emerges naturally from the compatibility condition on the difference tensor $D$. The result provides a canonical geometric solution to the gauge equation, allowing us to build Koszul–Vinberg double complexes and spectral sequences with intrinsic geometric meaning. This places the coKähler structure alongside holomorphic statistical geometry as a setting where deep gauge-theoretic phenomena can be studied naturally.

\begin{defn}
A coKähler statistical structure on a smooth manifold $M^{2n+1}$ is a quadruple $(\nabla, g, \varphi, \xi)$ such that:
\begin{enumerate}
    \item \((g, \varphi, \xi, \eta)\) defines a coKähler structure on \(M\), with \(\eta = g(\cdot, \xi)\),
    \item \(\nabla\) is a torsion-free affine connection, and the pair \((\nabla, g)\) defines a statistical structure (i.e., there exists a dual torsion-free connection \(\nabla^*\) such that
    \[
    X \cdot g(Y, Z) = g(\nabla_X Y, Z) + g(Y, \nabla^*_X Z)),
    \]
    \item The difference tensor \(D = \nabla - \nabla^{g}\) satisfies the compatibility condition:
    \[
    D(X, \varphi Y) + \varphi D(X, Y) = 0,
    \]
    for all vector fields \(X, Y\).
\end{enumerate}
\end{defn}
\begin{prop}
Let \((\nabla, g, \varphi, \xi)\) be a statistical coKähler structure on a manifold \(M\).
Then the structure tensor \(\varphi\) satisfies the gauge equation:
\[
\nabla_X \varphi = \varphi \nabla^*_X \quad \text{for all } X \in TM.
\]
\end{prop}
\begin{proof}
Let \(D = \nabla - \nabla^{g}\) be the difference tensor between the affine connection \(\nabla\) and the Levi-Civita connection \(\nabla^{g}\), and let \(\nabla^* = 2\nabla^{g} - \nabla\) be the dual connection with respect to \(g\). Then for any vector fields \(X, Y\), we compute:
\[
\begin{aligned}
\nabla_X(\varphi Y) - \varphi(\nabla^*_X Y)
&= \left[\nabla^{g}_X(\varphi Y) + D(X, \varphi Y)\right] - \varphi\left[\nabla^{g}_X Y - D(X, Y)\right] \\
&= (\nabla^{g}_X \varphi)Y + D(X, \varphi Y) + \varphi D(X, Y).
\end{aligned}
\]
Since the coKähler structure satisfies \(\nabla^{g} \varphi = 0\), and by assumption \(D(X, \varphi Y) + \varphi D(X, Y) = 0\), we conclude:
\[
\nabla_X(\varphi Y) = \varphi(\nabla^*_X Y),
\]
which is the gauge equation.
\end{proof}
\begin{prop}
Let \((\nabla, g, \varphi, \xi)\) be a statistical coKähler structure on a manifold \(M\).
Then, the tensor expression
\[
-\varphi^* D(\varphi X, Y) + \varphi D(X, \varphi^* Y)
\]
vanishes identically for all vector fields \(X, Y\), where \(\varphi^* = -\varphi\) is the \(g\)-adjoint of \(\varphi\). In particular, the corresponding Nijenhuis-type tensor from Remark 3.4 vanishes identically.
\end{prop}
\begin{rem}
This result shows that the coKähler structure tensor \(\varphi\) is a canonical gauge-compatible tensor in the sense of Section 3. Its normality and skew-symmetry, together with the statistical compatibility condition on \(D\), ensure the vanishing of the associated Nijenhuis-type tensor. This places \(\varphi\) on the same footing as complex structures in holomorphic statistical manifolds, providing a natural source of geometric gauge solutions and structured double complexes.
\end{rem}
On a $(2n+1)$-dimensional coKähler manifold 
\((M,\varphi,\xi,\eta,g)\), there is the canonical orthogonal splitting
\[
TM = \mathcal{D} \oplus \langle \xi \rangle,
\qquad \mathcal{D} = \ker \eta,
\]
where $\mathcal{D}$ is the $2n$-dimensional distribution orthogonal to the Reeb vector field $\xi$.
In this adapted decomposition, the structure tensor $\varphi$ satisfies
\[
\varphi(\xi) = 0, \qquad \varphi^2|_{\mathcal{D}} = -\mathrm{Id}_{\mathcal{D}},
\]
so that $\varphi$ induces an almost complex structure on $\mathcal{D}$.

We use a local $g$--orthonormal frame adapted to the splitting 
$TM = \mathcal{D} \oplus \langle \xi \rangle$:
\[
\{\xi, E_1, \dots, E_{2n}\}, \qquad E_{2k} = J E_{2k-1}, \quad k=1,\dots,n,
\]
so that any vector $v \in T_p M$ decomposes as
\[
v = \big(v_{\mathcal{D}}, v_{\xi}\big),
\]
with $v_{\mathcal{D}} \in \mathcal{D}_p$ and $v_{\xi} \in \langle \xi_p \rangle$.

In block form we may write
\[
\varphi = 
\begin{pmatrix}
J & 0 \\[4pt]
0 & 0
\end{pmatrix},
\qquad 
\text{acting on }
\begin{pmatrix}
v_{\mathcal{D}} \\[2pt]
v_{\xi}
\end{pmatrix},
\]
where $J^2 = -\mathrm{Id}_{\mathcal{D}}$ and $v_{\mathcal{D}} \in \mathcal{D}$, 
$v_{\xi} \in \langle \xi \rangle$.

We can write a general $(1,1)$--tensor $\theta$ (candidate gauge solution) 
in block form with respect to the orthogonal splitting 
$TM = \mathcal{D} \oplus \langle \xi \rangle$ as
\[
\theta \;=\; 
\begin{pmatrix}
B & u \\[4pt]
v^{\top} & a
\end{pmatrix},
\]
where:
\begin{itemize}
    \item $B : \mathcal{D} \to \mathcal{D}$ is the restriction of $\theta$ to $\mathcal{D}$
    (a $2n\times 2n$ matrix in a chosen basis),
    \item $u : \langle \xi \rangle \to \mathcal{D}$ is given by $u = \pi_{\mathcal{D}} \theta(\xi)$,
    i.e.\ a column vector in $\mathcal{D}$ representing the $\mathcal{D}$--component of $\theta(\xi)$,
    \item $v^{\top} : \mathcal{D} \to \langle \xi \rangle$ is the linear functional 
    $v(Y) = g(\theta(Y), \xi)$, i.e.\ a row vector representing the $\langle \xi \rangle$--component 
    of $\theta$ applied to $\mathcal{D}$,
    \item $a \in C^{\infty}(M)$ is the scalar component of $\theta(\xi)$ along~$\xi$.
\end{itemize}
The commutator of $\theta$ and $\varphi$ is
\[
[\theta,\varphi] 
= \theta\varphi - \varphi\theta
= \begin{pmatrix}
B J - J B & - J u \\[4pt]
v^{\top} J & 0
\end{pmatrix}.
\]
Thus $[\theta,\varphi] \neq 0$ may arise in three distinct ways:
\begin{enumerate}
    \item The $\mathcal{D}$--component $B$ does not commute with $J$, i.e.\ $[B,J] \neq 0$.
    \item The vector $u$ is nonzero (mixing $\xi \to \mathcal{D}$): the upper-right block $-J u$ is nonzero.
    \item The covector $v$ is nonzero (mixing $\mathcal{D} \to \xi$): the lower-left block $v^{\top} J$ is nonzero.
\end{enumerate}
For each fixed $X \in T_p M$, the map 
\[
Y \longmapsto D(X,Y)
\]
is a linear endomorphism of $T_p M$.  
Thus, for each $X$ we can represent $D(X)$ as a block matrix
\[
D(X) \;\equiv\;
\begin{pmatrix}
D_{11}(X) & D_{12}(X) \\[4pt]
D_{21}(X) & D_{22}(X)
\end{pmatrix},
\]
where the blocks correspond to the orthogonal splitting 
$T_p M = \mathcal{D}_p \oplus \langle \xi_p \rangle$.

On the other hand, the gauge equation is equivalent to
\begin{equation}
(\nabla^g_X \theta)(Y) + D(X,\theta Y) + \theta\big( D(X,Y) \big) = 0,
\qquad \forall X,Y \in TM.
\tag{$\star$}
\end{equation}
The equation \((\star)\) becomes, for each $X \in TM$,
\begin{equation}
(\nabla^g_X \theta) \;+\; D(X)\,\theta \;+\; \theta\,D(X) \;=\; 0.
\tag{1}
\end{equation}
Expanding \((1)\) in block form gives the following system of equations:

\begin{align}
\text{(1a)} &:\quad 
(\nabla^g_X B) + D_{11}(X) B + D_{12}(X) v^\top + B D_{11}(X) + u D_{21}(X) = 0,
\nonumber \\[4pt]
\text{(1b)} &:\quad 
(\nabla^g_X u) + D_{11}(X) u + D_{12}(X) a + B D_{12}(X) + u D_{22}(X) = 0,
\nonumber \\[4pt]
\text{(1c)} &:\quad 
(\nabla^g_X v^\top) + D_{21}(X) B + D_{22}(X) v^\top + v^\top D_{11}(X) + a D_{21}(X) = 0,
\nonumber \\[4pt]
\text{(1d)} &:\quad 
X(a) + D_{21}(X) u + v^\top D_{12}(X) + 2 a \, D_{22}(X) = 0.
\nonumber
\end{align}
\begin{lem}
Let $(M,\phi,\xi,\eta,g)$ be a statistical coK\"ahler manifold. 
With respect to the orthogonal splitting $TM = D \oplus \langle\xi\rangle$ and the block form
\[
\phi =
\begin{pmatrix}
J & 0\\
0 & 0
\end{pmatrix},
\]
write $D(X)$ as
\[
D(X) =
\begin{pmatrix}
D_{11}(X) & D_{12}(X)\\
D_{21}(X) & D_{22}(X)
\end{pmatrix}.
\]
Then, for every $X \in TM$,
\[
D_{12}(X) = 0, \qquad D_{21}(X) = 0, \qquad D_{11}(X)J + J\,D_{11}(X) = 0,
\]
and $D_{22}(X)$ is a scalar multiple of the identity on $\langle\xi\rangle$.
\end{lem}

\begin{proof}
First take $Y = \xi$. Since $\phi \xi = 0$, the relation $D(X,\phi Y) + \phi D(X,Y) = 0$ reduces to $\phi D(X,\xi) = 0$. This means $D(X,\xi) \in \langle\xi\rangle$, hence $D_{12}(X) = 0$.

Next, take $Y \in D$. Then $\phi Y = JY \in D$. The $\langle\xi\rangle$--component of $D(X,\phi Y) + \phi D(X,Y) = 0$ is
\[
D_{21}(X) (JY) + 0 = 0.
\]
Since $J$ is invertible on $D$, this implies $D_{21}(X) \equiv 0$.  
The $D$--component of the same relation reads
\[
D_{11}(X) JY + J D_{11}(X) Y = 0,
\]
for all $Y \in D$, i.e.
\[
D_{11}(X) J + J\,D_{11}(X) = 0.
\]
Finally, since $\langle\xi\rangle$ is one--dimensional, $D_{22}(X)$ is necessarily a scalar. 
\end{proof}

With the above simplifications, the gauge equation $\nabla \theta = \theta \nabla^\ast$ in block form
\[
\theta =
\begin{pmatrix}
B & u\\
v^{\top} & a
\end{pmatrix}
\]
reduces to the following system: for all $X \in TM$,
\begin{align}
(\nabla^g_X B) + D_{11}(X) B + B\,D_{11}(X) &= 0, \tag{1a'}\\
(\nabla^g_X u) + D_{11}(X) u + u\,D_{22}(X) &= 0, \tag{1b'}\\
(\nabla^g_X v^{\top}) + v^{\top} D_{11}(X) + D_{22}(X)\,v^{\top} &= 0, \tag{1c'}\\
X(a) + 2 a\, D_{22}(X) &= 0. \tag{1d'}
\end{align}


Fix a vector field $X$ and an integral curve $\gamma(s)$ with $\dot{\gamma}(s) = X(\gamma(s))$. 
Let $\nabla^g_{\dot{\gamma}}$ denote the covariant derivative along $\gamma$. 
Define
\begin{equation}
\notag
\delta(s) := \int_{0}^{s} D_{22}\!\big(X\big)\!\left(\gamma(\tau)\right)\,d\tau,
\qquad
\mathcal{P}_{11}(s) := \mathcal{P}\exp\!\left(-\int_{0}^{s} D_{11}\!\big(X\big)\!\left(\gamma(\tau)\right)\,d\tau\right),
\end{equation}
where $\mathcal{P}\exp$ is the path--ordered exponential.

If $B_0, u_0, v_0^\top, a_0$ denote the initial values of the components of $\theta$ at $s=0$, the unique solution of \((1a')\)–\((1d')\) along $\gamma$ is given by
\begin{align}
B(s) &= \mathcal{P}_{11}(s)\,B_0\,\mathcal{P}_{11}(s), \tag{A} \\
u(s) &= e^{-\delta(s)}\,\mathcal{P}_{11}(s)\,u_0, \tag{B} \\
v^\top(s) &= e^{-\delta(s)}\,v_0^\top\,\mathcal{P}_{11}(s), \tag{C} \\
a(s) &= a_0\,e^{-2\delta(s)}. \tag{D}
\end{align}
\begin{lem}
Let $(M,\varphi,\xi,\eta,g,\nabla)$ be a statistical coK\"ahler manifold and
assume the hypotheses of Lemma 4.3. Along an integral curve $\gamma(s)$ of a vector field $X=\dot\gamma(s)$ the block equation
\begin{equation}\tag{$1a'$}
\nabla^g_X B + D_{11}(X)B + B\,D_{11}(X) = 0
\end{equation}
implies the following.

\begin{enumerate}
\item The endomorphism $BJ$ satisfies the linear ODE
\[
\nabla^g_X(BJ) + [D_{11}(X),\,BJ] = 0,
\]
hence, if we set
\[
U(s) \;=\; \mathcal P\!\exp\!\Big(-\int_0^s D_{11}(\dot\gamma(\tau))\,d\tau\Big),
\qquad U(0)=\mathrm{Id},
\]
we have the conjugation formula
\[
BJ(s) \;=\; U(s)\, (BJ)(0)\, U(s)^{-1}.
\]

\item The endomorphism $JB$ satisfies the linear ODE
\[
\nabla^g_X(JB) + [JB,\,D_{11}(X)] = 0,
\]
and with $U(s)$ as above one obtains
\[
JB(s) \;=\; U(s)^{-1}\, (JB)(0)\, U(s).
\]

\item Let $C(s):=[B(s),J]=B(s)J-JB(s)$. Suppose that at $s=0$ the pair
$(BJ(0),\,JB(0))$ is not conjugate, i.e.\ there exists no invertible
$Q\in\mathrm{GL}(D)$ with
\[
BJ(0)=Q\,(JB(0))\,Q^{-1}.
\]
Then $C(s)\neq 0$ for all $s$ along $\gamma$; equivalently, if $[B(0),J]\neq0$
and $BJ(0)$ is not conjugate to $JB(0)$, the commutator $[B(s),J]$ never
vanishes along $\gamma$.
\end{enumerate}
\end{lem}

\begin{proof}
Multiply (1a') on the right by $J$. Using $\nabla^g J=0$ and the
anti-commutation $D_{11}J=-J D_{11}$ (Lemma 4.3) we get
\[
\nabla^g_X(BJ) + D_{11}(X)BJ - BJ\,D_{11}(X) = 0,
\]
which is exactly
\[
\nabla^g_X(BJ) + [D_{11}(X),\,BJ] = 0.
\]
Along the curve $\gamma$ this is an ordinary linear matrix ODE for the
time-dependent endomorphism $A(s):=BJ(s)$. The standard solution of
$\dot A + [D_{11}(s),A]=0$ is conjugation by the fundamental solution
$U(s)$ of $\dot U=-D_{11}(s)U$, $U(0)=\mathrm{Id}$; writing the solution
via the path-ordered exponential gives the formula
$BJ(s)=U(s)\,(BJ)(0)\,U(s)^{-1}$.

Similarly, multiply (1a') on the left by $J$ and use $J D_{11} = - D_{11} J$:
\[
\nabla^g_X(JB) - D_{11}(X)JB + JB\,D_{11}(X) = 0,
\]
i.e.
\[
\nabla^g_X(JB) + [JB,\,D_{11}(X)] = 0.
\]
If $U(s)$ is as above then $U(s)^{-1}$ satisfies $\dot W = D_{11}(s)W$,
$W(0)=\mathrm{Id}$, and one checks that
$JB(s)=U(s)^{-1}\,(JB)(0)\,U(s)$.

Finally, assume by contradiction that there exists $s_0$ with $C(s_0)=0$.
Then $BJ(s_0)=JB(s_0)$. Using the conjugation formulas we obtain
\[
U(s_0)\,(BJ)(0)\,U(s_0)^{-1} \;=\; U(s_0)^{-1}\,(JB)(0)\,U(s_0).
\]
Multiply on the left by $U(s_0)^{-1}$ and on the right by $U(s_0)^{-1}$ to get
\[
(BJ)(0) \;=\; U(s_0)^{-2}\,(JB)(0)\,U(s_0)^{2},
\]
so $(BJ)(0)$ and $(JB)(0)$ are conjugate (by $Q=U(s_0)^{-2}$). This contradicts
the hypothesis that they are not conjugate. Hence $C(s)\neq0$ for all $s$,
which proves persistence of non-commutativity along $\gamma$.
\end{proof}























