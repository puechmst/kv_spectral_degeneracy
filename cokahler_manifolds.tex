\section{coK\"{a}hler manifolds}
CoK\"{a}hler manifolds form a class of almost contact metric manifolds closely related to both K\"{a}hler and cosymplectic geometry. Let $(M^{2n+1}, \phi, \xi, \eta, g)$ be an almost contact metric manifold, where:
\begin{enumerate}
\item $\phi$ is a (1,1)-tensor field,
\item $\xi$ is the Reeb (characteristic) vector field,
\item $\eta$ is the contact 1-form,
\item $g$ is a Riemannian metric satisfying:

  $$
  \eta(\xi) = 1,\quad \phi^2 = -\mathrm{Id} + \eta \otimes \xi,\quad g(\phi X, \phi Y) = g(X, Y) - \eta(X)\eta(Y).
  $$
\end{enumerate}
A coK\"{a}hler structure is defined by the following conditions:
\begin{enumerate}
\item $\nabla \phi = 0$,
\item $\nabla \eta = 0$, and
\item $d\eta = 0$, $d\Phi = 0$, where $\Phi(X,Y) = g(X, \phi Y)$.
\end{enumerate}
These conditions imply that $\phi$, $\eta$, and $g$ are all parallel with respect to the Levi-Civita connection $\nabla$, and that $(M, \phi, \xi, \eta, g)$ is both normal and cosymplectic. Importantly, the almost contact structure induces a foliation by leaves orthogonal to $\xi$, each of which inherits a K\"{a}hler structure from $\phi$ and $g$.

This leads to a canonical gauge structure on coK\"{a}hler manifolds, making them suitable for the application of KV-algebra techniques. Specifically, the Levi-Civita connection $\nabla$ on a coK\"{a}hler manifold is torsion-free and preserves the underlying structure tensors, allowing one to define pre-Lie products via

$$
X \circ Y := \nabla_X Y
$$

for vector fields $X, Y \in \Gamma(TM)$. The curvature conditions further imply symmetries in the associator that reflect pre-Lie algebra identities.

In particular, the canonical splitting $TM = \mathcal{D} \oplus \langle \xi \rangle$, where $\mathcal{D} = \ker(\eta)$, allows one to study the restriction of the Koszul-Vinberg algebra to the K\"{a}hler leaves and analyze the corresponding spectral sequence. These foliated structures are flat in transverse directions and lead naturally to vanishing torsion tensors $T_\theta$ when considering natural normal fields such as $\phi$. Thus, coK\"{a}hler manifolds provide an explicit class of geometric models where the degeneracy of the KV spectral sequence can be investigated under controlled curvature and holonomy assumptions.

Let $M = S^1 \times \mathbb{R}^{2n}$ with coordinates $(t, x^1, y^1, \dots, x^n, y^n)$. Define the flat product metric:

$$
g = dt^2 + \sum_{j=1}^n \left( (dx^j)^2 + (dy^j)^2 \right)
$$

Define a coK\"{a}hler structure:
\begin{itemize}
\item $\xi = \partial_t;$
\item $\eta = dt;$
\item $\phi(\partial_{x^j}) = \partial_{y^j}, \; \phi(\partial_{y^j}) = -\partial_{x^j}, \; \phi(\partial_t) = 0.$
\end{itemize}


Let $\nabla$ be the flat connection. Define the dual connection $\nabla^*$ as:

$$
\nabla^*_X Y = \nabla_X Y + D(X, Y)
$$

where $D$ is a symmetric (1,2)-tensor field such that:

$$
g(D(X,Y),Z) = -g(Y,D(X,Z)).
$$


Define $\theta$ diagonally:

$$
\theta = \lambda_0 \, \partial_t \otimes dt + \sum_{j=1}^n \left( \lambda_j \, \partial_{x^j} \otimes dx^j + \mu_j \, \partial_{y^j} \otimes dy^j \right)
$$

This $\theta$ is symmetric by construction.


Given $\nabla \theta = 0$, the gauge equation reduces to:

$$
0 = \theta D(X, \cdot).
$$
\begin{ex}
In dimension 2, let $A(\partial_x, \partial_x) = \partial_y$. Then $\theta(\partial_y) = 0$ is required. Thus, $\mu = 0$ is forced.
\end{ex}
\begin{ex}
Let $(M, \varphi, \xi, \eta, g)$ be a flat coKähler manifold with local orthonormal frame $\{\xi, E_2, \ldots, E_n\}$ satisfying $\eta(\xi) = 1$, $\eta(E_j) = 0$, and let $\nabla$ denote the flat, torsion-free Levi-Civita connection. Define the symmetric $(1,1)$-tensor
\[
\theta = a(t)\, \eta \otimes \xi + \sum_{j=2}^n b_j(t)\, E^j \otimes E_j,
\]
where $E^j$ are the dual $1$-forms of $E_j$, and $\theta$ is time-dependent only. Assume $E_j$ are parallel along $\xi$, i.e., $\nabla_\xi E_j = 0$.

We verify the gauge-type condition
\[
(\nabla_X \theta)(Y) = \eta(Y)\theta(X) + \eta(X)\theta(Y),
\]
for all vector fields $X, Y$ on $M$.

\begin{itemize}
\item For $X = Y = \xi$:
\[
(\nabla_\xi \theta)(\xi) = \nabla_\xi(\theta(\xi)) - \theta(\nabla_\xi \xi) = \nabla_\xi(a(t)\xi) = a'(t)\xi.
\]
On the other hand, the right-hand side gives $2a(t)\xi$. Therefore,
\[
a'(t) = 2a(t) \quad \Rightarrow \quad a(t) = A e^{2t}.
\]

\item For $X = \xi$, $Y = E_j$:
\[
(\nabla_\xi \theta)(E_j) = \nabla_\xi(b_j(t) E_j) = b_j'(t) E_j,
\]
\[
\eta(E_j) = 0, \quad \eta(\xi) = 1 \Rightarrow \eta(Y)\theta(\xi) + \eta(X)\theta(Y) = \theta(E_j).
\]
Thus,
\[
b_j'(t) = b_j(t) \quad \Rightarrow \quad b_j(t) = B_j e^t.
\]

\item For $X = E_k$, $Y$ arbitrary:
The gauge condition becomes:
\[
(\nabla_{E_k} \theta)(Y) = \eta(Y)\theta(E_k).
\]
Compute the left-hand side:
\[
(\nabla_{E_k} \theta)(Y) = \nabla_{E_k}(\theta(Y)) - \theta(\nabla_{E_k} Y).
\]
For $Y = \xi$:
\[
(\nabla_{E_k} \theta)(\xi) = E_k(a)\xi + a \nabla_{E_k} \xi - \theta(\nabla_{E_k} \xi).
\]
Equating both sides:
\[
E_k(a)\xi + (a - \theta)\nabla_{E_k} \xi = \theta(E_k).
\]
Thus, either $\nabla_{E_k} \xi = 0$ and $E_k(a) = 0$ (implying $\theta(E_k) = 0$), or $\theta$ must be chosen to satisfy:
\[
\theta(E_k) = E_k(a)\xi + (a - \theta)(\nabla_{E_k} \xi).
\]
\end{itemize}

This example demonstrates how exponential time-dependent components of $\theta$ can satisfy the gauge-type condition on flat coKähler manifolds under appropriate geometric constraints.
\end{ex}
\begin{ex}

We now present a nontrivial example where the tensor field $\theta$ depends polynomially on the coordinate $t$ and satisfies the gauge equation $\nabla \theta = \theta \nabla^*$, but only after modifying the background geometry. This illustrates the necessity of adjusting the difference tensor $D$ when seeking polynomial solutions in $\theta$.

Let $M = S^1 \times \mathbb{R}^{2n}$ with coordinates $(t, x_1, y_1, \ldots, x_n, y_n)$ and flat coKähler structure as in previous examples. Define the symmetric $(1,1)$-tensor field
\[
\theta = a(t)\, \eta \otimes \xi + \sum_{j=1}^n \left( b_j(t)\, dx_j \otimes \partial_{x_j} + c_j(t)\, dy_j \otimes \partial_{y_j} \right),
\]
where the coefficient functions are taken to be polynomials in $t$, e.g., $a(t) = A t^k$, $b_j(t) = B_j t^{\ell_j}$, $c_j(t) = C_j t^{m_j}$ for constants $A, B_j, C_j$ and non-negative integers $k, \ell_j, m_j$.

Assume the connection $\nabla$ is flat and torsion-free, so that $\nabla \theta = \partial_t \theta$. In the previous examples, a constant $\theta$ sufficed to make $\nabla \theta = 0$ and the gauge equation reduced to $\theta D = 0$. However, for this variable $\theta$, the equation $\nabla \theta = \theta D$ becomes
\[
\frac{d}{dt} \theta = \theta D(\partial_t, \cdot),
\]
which imposes strong constraints on $D$.

To satisfy this equation, we define a non-flat difference tensor:
\[
D(X,Y) := \eta(X)\, \varphi(Y) + t \cdot \varphi(X)\, \eta(Y).
\]
We compute:
\[
\partial_t b_j(t) = t \cdot b_j(t) \quad \Rightarrow \quad b_j(t) = B_j \exp\left(\frac{t^2}{2}\right),
\]
so true polynomial solutions do not arise. However, for $b_j(t) = B_j t$, we find that:
\[
\partial_t b_j(t) = B_j = t \cdot b_j(t) \quad \Leftrightarrow \quad t = 1,
\]
which is only valid on a slice of $M$. Thus, a  polynomial solution is only compatible with the gauge equation if $D$ is choosen. 
\end{ex}

