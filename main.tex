\documentclass{amsart}
\usepackage[utf8]{inputenc}
\usepackage{amssymb,amsmath,amsfonts,amsthm}
\usepackage{graphicx}
\usepackage{stmaryrd}
\usepackage[utf8]{inputenc}
\usepackage[T1]{fontenc}
\usepackage{tikz}
\usetikzlibrary{positioning}
\usepackage{tikz-cd}
\usepackage{enumitem}

\newcommand{\R}{\ensuremath{\mathbb{R}}}
\newcommand{\C}{\ensuremath{\mathbb{C}}}
\newcommand{\K}{\ensuremath{\mathbb{K}}}
\newcommand{\Q}{\ensuremath{\mathbb{Q}}}
\newcommand{\N}{\ensuremath{\mathbb{N}}}
\newcommand{\Z}{\ensuremath{\mathbb{Z}}}
\newcommand{\usph}[1]{\ensuremath{\mathbb{S}^{#1}}}
\newcommand{\Aff}{\ensuremath{\text{Aff}}}
\newcommand{\aff}{\ensuremath{\mathfrak{aff}}}
\newcommand{\gl}{\ensuremath{\mathfrak{gl}}}
\newcommand{\GL}{\ensuremath{\text{GL}}}
\newcommand{\homfunc}[2]{\ensuremath{\text{Hom}\left(#1,#2\right)}}
\newcommand{\comp}[2]{\ensuremath{\text{H}\left(#1,#2\right)}}
\newcommand{\homology}[3]{\ensuremath{\text{H}^{#1}\left(#2,#3\right)}}
\newcommand{\frakg}{\ensuremath{\mathfrak{g}}}
\newcommand{\pairc}[2]{\ensuremath{\left\langle #1,#2 \right\rangle_+}}
\newcommand{\paircs}[2]{\ensuremath{\left\langle #1,#2 \right\rangle_-}}
\newcommand{\pairct}[2]{\ensuremath{\left\langle #1,#2 \right\rangle_\theta}}
\newcommand{\cbracket}[2]{\ensuremath{\left\llbracket #1,#2 \right\rrbracket_c}}
\newcommand{\dbracket}[2]{\ensuremath{\left\llbracket #1,#2 \right\rrbracket_d}}
\newcommand{\lieder}[2]{\ensuremath{\mathcal{L}_#1 #2}}
\newcommand{\parder}[2]{\ensuremath{\frac{\partial #1}{\partial #2}}}
\newcommand{\gmet}[2]{\ensuremath{g\left(#1, #2  \right)}}
\newcommand{\pullg}[2]{\ensuremath{\tilde{g}\left(#1, #2  \right)}}
\newcommand{\nnet}[2]{\ensuremath{\mathcal{N}\left( #1,#2 \right)}}
\newcommand{\mnet}[1]{\ensuremath{\mathcal{N}_{W}\left( #1\right)}}
\newcommand{\lc}{\ensuremath{\nabla^{\text{lc}}}}
\newcommand{\fnet}{\ensuremath{\mathcal{N}_W}}
\newcommand{\kldiv}[2]{\ensuremath{\textit{KL}\left( #1,#2 \right)}}
\newcommand{\gaugecat}{\ensuremath{\mathcal{GC}}}
\newcommand{\gaugecatu}{\ensuremath{\mathcal{GU}}}
\DeclareMathOperator{\im}{\ensuremath{\textsf{im}}}
\DeclareMathOperator{\ad}{ad}
\DeclareMathOperator{\trace}{\ensuremath{\text{Tr}}}
\DeclareMathOperator{\ext}{\ensuremath{\text{Ext}}}

\theoremstyle{plain}
\newtheorem{thm}{Theorem}[section]
\newtheorem{lem}[thm]{Lemma}
\newtheorem{prop}[thm]{Proposition}
\newtheorem{cor}[thm]{Corollary}

\theoremstyle{remark}
\newtheorem{rem}{Remark}[section]
\newtheorem{ex}{Example}[section]
\newtheorem{notation}{Example}[section]

\theoremstyle{definition}
\newtheorem{defn}{Definition}[section]
\title{Degeneracy of Koszul-Vinberg spectral sequence.}
\author{Mirjana Milijevic}
\address{Faculty of Economics, University of Banja Luka}
\email{mirjana.milijevic@ef.unibl.org}
\author{Stéphane PUECHMOREL}

\address{ENAC- Université de Toulouse, 7, avenue Edouard Belin, 31055 Toulouse cedex }
\email{stephane.puechmorel@enac.fr}

\date{June 2025}

\begin{document}

\maketitle

\section{Introduction}
Koszul–Vinberg algebras (also known as pre-Lie algebras) naturally arise in differential geometry through the use of torsion-free affine connections. When equipped with a gauge structure $(M, \nabla)$, the space of vector fields on a smooth manifold inherits a pre-Lie algebra structure defined by the product $X \circ Y = \nabla_X Y$. Unlike Lie algebras, this structure does not satisfy the Jacobi identity, but instead encodes curvature information through the associator.

In this paper, we study the interaction between pre-Lie algebras, gauge-theoretic structures, and symmetric $(1,1)$-tensor fields that satisfy a compatibility condition known as the gauge equation:

$$
\nabla \theta = \theta \nabla^*,
$$

where $\nabla$ and $\nabla^*$ are torsion-free affine connections dual with respect to a Riemannian metric $g$. This equation governs how $\theta$ mediates between two flat (or nearly flat) geometric structures and plays a central role in the construction of differential complexes arising from Koszul–Vinberg (KV) algebras. Furthermore, we study a distinct KV-algebra structure defined not by the connection $\nabla$, but by a symmetric (1,2)-tensor field $D$, defined as the difference between two torsion-free connections: $D := \nabla^* - \nabla$. We consider the product:

$X \circ Y := D(X, Y),$

which is symmetric and defines a valid pre-Lie algebra when $D$ satisfies certain compatibility conditions. In particular, we explore the associated cohomological structure and spectral sequence induced by this KV product.

The central goal is to identify conditions under which the spectral sequence associated to the cohomology of this KV algebra degenerates early. We show that when a symmetric tensor field $\theta$ satisfies a gauge-type condition adapted to $D$, the resulting spectral sequence degenerates at a low stage.

We work within the setting of flat coKähler manifolds, where the underlying geometric structures—such as the Reeb field $\xi$, the contact form $\eta$, and the endomorphism $\phi$—induce a natural splitting of the tangent bundle and allow for explicit computations. In this context, we reinterpret classical gauge equations and deformation tensors relative to the new KV-algebra structure based on $D$, not $\nabla$.

The paper is organized as follows: Section 2 introduces the general theory of KV algebras and their cohomology, adapted to the case where the product is defined by a symmetric tensor $D$. Section 3 introduces a generalized gauge equation and explores conditions under which cochains and higher-order obstructions vanish. Section 4 focuses on flat coKähler manifolds and presents explicit computations showing the degeneracy of the spectral sequence associated to the $D$-based KV algebra.

This reinterpretation of KV-algebra structures and their cohomology reveals a new class of degeneracy results governed by the interplay between geometry (via the coKähler structure), algebra (via $D$), and analysis (via the gauge tensor $\theta$).

This work contributes to the understanding of the geometric meaning behind the gauge equation in differential geometry and its algebraic consequences in the deformation theory and cohomology of pre-Lie algebras.

\section{Koszul-Vinberg algebras}
Koszul-Vinberg algebras, more commonly known as pre-Lie algebras, 
arise naturally in differential geometry. Let $\left(M ,\nabla \right)$ be a gauge structure with $\nabla$ a torsionless Koszul connection on $M$. A product is defined on $\chi\left(M\right)=\Gamma\left(TM \right)$ by:
\begin{equation}
    \label{eq:example_kv}
    X \circ Y = \nabla_X Y, \, X,Y \in \chi\left(M\right).
\end{equation}
The commutator $X\circ Y - Y \circ X$ is the usual Lie bracket, but the Jacobi identity is not satisfied by the product $\circ$. The associator is given by the relation:
\begin{equation}
    \label{eq:associator_nabla}
    \left( X, Y, Z \right) = \nabla_{\nabla_X Y} Z - \nabla_X \nabla_Y Z = - \nabla^2_{X,Y}Z, \, X, Y, Z \in \chi\left(M\right).
\end{equation}
From the above property, it is easy to derive the next proposition:
\begin{prop}
\label{prop:kv_defect}
For any $X,Y,Z \in \chi\left(M\right):$
\begin{equation}
\label{eq:kv_defect}
    \left( X, Y, Z \right) - \left( Y, X, Z \right) = - R^\nabla\left( X, Y \right) Z,
\end{equation}
with $R^\nabla$ the curvature of $\nabla.$
\end{prop}
\begin{proof}
Obvious from the expression: $R^\nabla\left( X, Y \right) Z = \nabla^2_{X,Y}Z - \nabla^2_{Y,X}Z.$
\end{proof}
If $\nabla$ is flat, then:
$\left( X, Y , Z \right) = \left( Y, X, Z \right),$ that is the defining property of a pre-Lie algebra.
\subsection{Definitions}
Let $k$ be a field, that will always be $\R$ or $\C$ in the following, and let $A$ be a $k$-vector space. 
\begin{defn}
\label{def:kv_def}
$A$ is said to be a pre-Lie (or a KV-algebra) if equipped with a product $\circ \colon A \times A \to A$ such that, for any $a,b,c$ in $A$:
\begin{equation}
    \label{eq:kv_def}
    (a,b,c)=(b,a,c), \, (a,b,c)=(a\circ b )\circ c - a \circ (b \circ c).
\end{equation}
\end{defn} 
\begin{prop}
\label{prop:subjacent_lie}
Let $A$ be a KV algebra. The bracket:
\begin{equation}
    \label{eq:bracket}
    \left[a,b\right] = a \circ b - b \circ a.
\end{equation}
provides $A$ with a Lie algebra structure, denoted by $A^{L}.$
\end{prop}
\begin{defn}
\label{def:lproduct}
Let $a \in A.$ The left product endomorphism $L_a$ is defined by:
\begin{equation}
    \label{eq:la}
    L_a b = a \circ b, \, b \in A.
\end{equation}
\end{defn}
\begin{prop}
\label{prop:lie_morphism}
For any $a,b, \in A:$
\begin{equation}
    \label{eq:lie_morphism}
    \left[ L_a, L_b \right] = L_{[a,b]}.
\end{equation}
\end{prop}
\begin{proof}
    Let $w \in A$. Then:
    \begin{equation}
    \left[ L_a, L_b \right]w = a \circ \left( b \circ w \right) - b \circ \left(a \circ w \right).
    \end{equation}
    Using the associator, the right hand term can be rewritten as:
    \begin{equation}
        \left( a \circ b \right) \circ w - \left( a,b,w \right) - \left( b \circ a \right) \circ w + \left( b,a,w \right) = \left( [a,b] \right)\circ w = L_{[a,b]} w,
    \end{equation}
   which concludes the proof. 
\end{proof}
The endomorphism $L_a$ extends to a derivation of the symmetric algebra $S(A)$ by application of the Liebniz rule:
\begin{equation}
    \label{eq:der_la}
    L_a \left( a_1 \otimes \dots \otimes a_q \right)= \sum_{i=1}^q a_1 \otimes \dots a a_i \dots \otimes a_q,
\end{equation}
hence there is a Lie algebra morphism $L \colon a \in A \mapsto \text{Der}S\left( A \right)$ by equation \ref{eq:lie_morphism}.
Now, for any $a \in A$, define:
\begin{equation}
    \label{eq:theta_morphism}
    \Theta_a x = a. x + L_a x, x \in S\left( A \right).
\end{equation}
\begin{prop}
    \label{prop:theta_morphism}
    $\Theta$ is a Lie algebra morphism from $A$ to $\text{End}(A).$
\end{prop}
\begin{proof}
    By a brute force approach, and since $S(A)$ is commutative:
    \begin{equation}
        \begin{split}
            & \left[ \Theta_a, \Theta_b \right]  x = a .  \Theta_b x + L_a \Theta_b x - b .  \Theta_a x - L_b \Theta_a x \\
            &= a . b . x + a . L_b x + L_a (b. x) + L_a L_b x - b . a . x - b L_a x - L_b (a . x) - L_b L_a x \\
            &= a . L_b x + b . L_a x + (a \circ b) . x +  L_a L_b x - b . L_a x - a . L_b x - (b \circ a). x - L_b L_a x \\
            &= \left( a \circ b - b \circ a \right) . x + \left[ L_a, L_b \right] x \\
            & = \Theta_{[a,b]} x.
        \end{split}
    \end{equation}
\end{proof}
$\Theta$ uniquely lifts to a Lie algebra morphism, still denoted by $\Theta$, between $\mathcal{U}(A)$, the universal envelopping algebra of $A$ and $S(A)$.
Finally, the morphism of graded algebra $ \theta \colon u \in \mathcal{U}(A) \to \Theta_u 1$ is an isomorphism since, for $u=u_1 \otimes \dots \otimes u_q$:
\begin{equation}
    \theta(u)=M_{u_1 \otimes \dots \otimes u_q} 1 = u_1 \otimes \dots \otimes u_q + r
\end{equation}
with $r$ of degree at most $q-1$, hence vanishing in the graded algebra. Gathering things, the next theorem is proved:
\begin{thm}[KV Poincaré-Birkhoff-Witt]
    \label{thm:pbw_prelie}
    Let $A$ be a pre-Lie algebra and let $\mathcal{U}(A)$ be the universal envelopping algebra of the Lie algebra associated to $A$. There exists an isomorphism 
$\theta \colon \mathcal{U}(A) \to S(A).$
\end{thm}
\begin{defn}[\cite{bai2008}]
Let $A$ be a KV algebra. A representation of $A$ on a vector space $V$ is a pair of linear maps $L,R$ from $A$ to $\mathfrak{GL}(V)$ such that $L$ is a Lie algebra morphism 
from $A^{L}$ to $\mathfrak{gl}(V)$ and $R$ satisfies:
\begin{equation}
    \label{eq:kv_representation}
    R(b)L(a)-R(a)L(b)=R(b)R(a)-R(a \circ b).
\end{equation}
A vector space $V$ equipped with a representation $L,R$ is said to be a $KV$-module (or pre-Lie module). 
\end{defn}
\begin{notation}
The action of $L$ (resp. $R$) on an element $v$ of $V$ is conveniently denoted as a left (resp. right) product:
\[
L(a)v = a \triangleleft v, \, R(a)v = v \triangleright a \, .
\]
\end{notation}
\begin{rem}
    Equation \ref{eq:kv_representation} is a weak form of associativity. In fact, the requirements on $L,R$ can be rewritten with associators as:
    \begin{equation}
        \label{eq:module_associator}
        \left( a, b, v \right) = \left( b, a , v \right), \, \left( a, v, b \right) = \left( v, a,b \right), \, a,b \in A, \, v \in V.
    \end{equation}
with:
\begin{equation}
    \begin{split}
    &\left( a,b,v \right) = \left( a \circ b \right) \triangleleft v - a \triangleleft b \triangleleft v, \\ 
    &\left( a, v, b \right) = \left( a \triangleleft v \right) \triangleright b - a \triangleleft \left( v \triangleright b \right) \\
    &\left( v, a, b \right) = \left( v \triangleright a \right) \triangleright b - v \triangleright \left( a \circ b \right).
    \end{split}
\end{equation}
\end{rem}
\subsection{Cohomology}




\section{The gauge equation}
\begin{prop}
    \label{prop:gen_nijenhuis}
    Let $\theta$ be normal and statisfying the gauge equation. Let $D=\nabla^*-\nabla$. Then the tensor:
    \begin{equation}
        \label{eq:gen_nijenhuis}
        \begin{split}
        T_\theta \colon (X,Y) \mapsto & \left[ \theta X, \theta^*Y \right] - \theta^* \left[ \theta X, Y \right] 
        - \theta \left[X, \theta^*Y  \right] + \theta^* \theta \left[ X,Y \right]\\
        & +\theta^* D\left( \theta X, Y \right) - \theta D\left( X, \theta^*Y \right)
        \end{split}
    \end{equation}
    vanishes identically.
\end{prop} 
\begin{proof}
    We must first prove the tensoriality. Let $f \colon M \to \R$ be a smooth function. Since $D$ is known to be 
    a tensor, there is no need to take the last two terms into account. A direct computation shows that:
    \begin{equation}
        T_\theta(fX,Y) = f T_\theta(X,Y) +
         \left( - \theta^* Y(f) + \theta^* Y(f) + \theta \theta^* Y(f) - \theta^* \theta Y(f) \right) T_\theta(X,Y)
    \end{equation}
    Since $\theta$ is normal, $ T_\theta(fX,Y) = f T_\theta(X,Y) $.
    Now, a similar computation yields:
    \begin{equation}
        T_\theta(X,fY) = f T_\theta(X,Y) + 
        \left( \theta X(f) - \theta^*\theta X(f) - \theta  X(f) + \theta^* \theta X(f) \right)T_\theta(X,Y) = 0
    \end{equation}
    proving that $T_\theta$ is a tensor.

    $\nabla^*$ being without torsion:
    \begin{equation}
        \begin{split}
            \left[ \theta X, \theta^* Y \right] & = \nabla^*_{\theta X} \theta^* Y - \nabla^*_{\theta^*Y} \theta X \\
            & = \theta^* \nabla_{\theta X} Y - \theta \nabla_{\theta^*Y} X \\
            & = \theta^* \left( \nabla_Y \theta X + \left[ \theta X, Y \right] \right) 
            - \theta \left( \nabla_X \theta^* Y  + \left[ \theta^*Y, X \right]\right) \\
        \end{split}
    \end{equation}
    Introducing the difference tensor $D \colon (X,Y) \mapsto \nabla_X^* Y - \nabla_X Y$:
    \begin{equation}
       \begin{split}
         \left[ \theta X, \theta^* Y \right]  =&
           \theta^* \theta \nabla^*_Y X - \theta \theta^* \nabla^*_X Y  - \theta^* D\left( \theta X, Y \right) + \theta D\left( X, \theta^*Y \right)\\
           & + \theta^* \left[ \theta X, Y \right] + \theta \left[ X, \theta^* Y \right]
       \end{split} 
    \end{equation}
    Since $\theta$ is normal, $\theta^* \theta = \theta \theta^*$, thus:
    \begin{equation}
        \left[ \theta X, \theta^* Y \right] = \theta^* \theta \left[ Y,X \right] + \theta^* \left[ \theta X, Y \right] + \theta \left[ X, \theta^* Y \right]
        -\theta^* D\left( \theta X, Y \right) + \theta D\left( X, \theta^*Y \right)
    \end{equation}
    and the claim follows.
\end{proof}
\begin{rem}
    The tensoriality of $T_\theta$ depends critically on the normality of $\theta$. This was already pointed out by 
    Nijenhuis.
\end{rem}
\begin{rem}
    $T_\theta$ is the Nijenhuis tensor when $\theta$ is skew-symmetric and $\nabla = \nabla^*$. When $\theta$ is normal and satisfies the gauge equation,
    the above results shows that $-\theta^*D\left(  \theta X, Y\right)+\theta D\left( X,\theta^*Y \right)$ is the Nijenhuis tensor. 
\end{rem}
\bibliographystyle{plain}
\bibliography{main}
\end{document}
