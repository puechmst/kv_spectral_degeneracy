\section{Koszul-Vinberg algebras}
Koszul-Vinberg algebras, more commonly known as pre-Lie algebras, 
arise naturally in differential geometry. Let $\left(M ,\nabla \right)$ be a gauge structure with $\nabla$ a torsionless Koszul connection on $M$. A product is defined on $\chi\left(M\right)=\Gamma\left(TM \right)$ by:
\begin{equation}
    \label{eq:example_kv}
    X \circ Y = \nabla_X Y, \, X,Y \in \chi\left(M\right).
\end{equation}
The commutator $X\circ Y - Y \circ X$ is the usual Lie bracket, but the Jacobi identity is not satisfied by the product $\circ$. The associator is given by the relation:
\begin{equation}
    \label{eq:associator_nabla}
    \left( X, Y, Z \right) = \nabla_{\nabla_X Y} Z - \nabla_X \nabla_Y Z = - \nabla^2_{X,Y}Z, \, X, Y, Z \in \chi\left(M\right).
\end{equation}
From the above property, it is easy to derive the next proposition:
\begin{prop}
\label{prop:kv_defect}
For any $X,Y,Z \in \chi\left(M\right):$
\begin{equation}
\label{eq:kv_defect}
    \left( X, Y, Z \right) - \left( Y, X, Z \right) = - R^\nabla\left( X, Y \right) Z,
\end{equation}
with $R^\nabla$ the curvature of $\nabla.$
\end{prop}
\begin{proof}
Obvious from the expression: $R^\nabla\left( X, Y \right) Z = \nabla^2_{X,Y}Z - \nabla^2_{Y,X}Z.$
\end{proof}
If $\nabla$ is flat, then:
$\left( X, Y , Z \right) = \left( Y, X, Z \right),$ that is the defining property of a pre-Lie algebra.
\subsection{Definition}
Let $k$ be a field, that will always be $\R$ or $\C$ in the following, and let $A$ be a $k$-vector space. 
\begin{defn}
\label{def:kv_def}
$A$ is said to be a pre-Lie (or a KV-algebra) if equipped with a product $\circ \colon A \times A \to A$ such that, for any $a,b,c$ in $A$:
\begin{equation}
    \label{eq:kv_def}
    (a,b,c)=(b,a,c), \, (a,b,c)=(a\circ b )\circ c - a \circ (b \circ c).
\end{equation}
\end{defn} 
\begin{defn}
\label{defn:bracket}
The bracket is defined, for any $a,b \in A,$ as:
\begin{equation}
    \label{eq:bracket}
    \left[a,b\right] = a \circ b - b \circ a?
\end{equation}
\end{defn}
\begin{defn}
\label{def:lproduct}
Let $a \in A.$ The left product endomorphism $L_a$ is defined by:
\begin{equation}
    \label{eq:la}
    L_a b = a \circ b, \, b \in A.
\end{equation}
\end{defn}
\begin{prop}
\label{prop:lie_morphism}
For any $a,b, \in A:$
\begin{equation}
    \label{eq:lie_morphism}
    \left[ L_a, L_b \right] = L_{[a,b]}.
\end{equation}
\end{prop}
\begin{proof}
    Let $w \in A$. Then:
    \begin{equation}
    \left[ L_a, L_b \right]w = a \circ \left( b \circ w \right) - b \circ \left(a \circ w \right).
    \end{equation}
    Using the associator, the right hand term can be rewritten as:
    \begin{equation}
        \left( a \circ b \right) \circ w - \left( a,b,w \right) - \left( b \circ a \right) \circ w + \left( b,a,w \right) = \left( [a,b] \right)\circ w = L_{[a,b]} w,
    \end{equation}
   which concludes the proof. 
\end{proof}
The endomorphism $L_a$ extends to a derivation of the tensor algebra $T(A)$ (resp. symmetric algebra $S(A)$) by application of the Liebniz rule:
\begin{equation}
    \label{eq:der_la}
    L_a \left( a_1 \otimes \dots \otimes a_q \right)= \sum_{i=1}^q a_1 \otimes \dots a a_i \dots \otimes a_q.
\end{equation}


